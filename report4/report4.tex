% Options for packages loaded elsewhere
\PassOptionsToPackage{unicode}{hyperref}
\PassOptionsToPackage{hyphens}{url}
%
\documentclass[
]{article}
\usepackage{lmodern}
\usepackage{amssymb,amsmath}
\usepackage{ifxetex,ifluatex}
\ifnum 0\ifxetex 1\fi\ifluatex 1\fi=0 % if pdftex
  \usepackage[T1]{fontenc}
  \usepackage[utf8]{inputenc}
  \usepackage{textcomp} % provide euro and other symbols
\else % if luatex or xetex
  \usepackage{unicode-math}
  \defaultfontfeatures{Scale=MatchLowercase}
  \defaultfontfeatures[\rmfamily]{Ligatures=TeX,Scale=1}
\fi
% Use upquote if available, for straight quotes in verbatim environments
\IfFileExists{upquote.sty}{\usepackage{upquote}}{}
\IfFileExists{microtype.sty}{% use microtype if available
  \usepackage[]{microtype}
  \UseMicrotypeSet[protrusion]{basicmath} % disable protrusion for tt fonts
}{}
\makeatletter
\@ifundefined{KOMAClassName}{% if non-KOMA class
  \IfFileExists{parskip.sty}{%
    \usepackage{parskip}
  }{% else
    \setlength{\parindent}{0pt}
    \setlength{\parskip}{6pt plus 2pt minus 1pt}}
}{% if KOMA class
  \KOMAoptions{parskip=half}}
\makeatother
\usepackage{xcolor}
\IfFileExists{xurl.sty}{\usepackage{xurl}}{} % add URL line breaks if available
\IfFileExists{bookmark.sty}{\usepackage{bookmark}}{\usepackage{hyperref}}
\hypersetup{
  pdftitle={Markovian Emotions},
  pdfauthor={Jack Lovell},
  hidelinks,
  pdfcreator={LaTeX via pandoc}}
\urlstyle{same} % disable monospaced font for URLs
\usepackage[margin=1in]{geometry}
\usepackage{color}
\usepackage{fancyvrb}
\newcommand{\VerbBar}{|}
\newcommand{\VERB}{\Verb[commandchars=\\\{\}]}
\DefineVerbatimEnvironment{Highlighting}{Verbatim}{commandchars=\\\{\}}
% Add ',fontsize=\small' for more characters per line
\usepackage{framed}
\definecolor{shadecolor}{RGB}{248,248,248}
\newenvironment{Shaded}{\begin{snugshade}}{\end{snugshade}}
\newcommand{\AlertTok}[1]{\textcolor[rgb]{0.94,0.16,0.16}{#1}}
\newcommand{\AnnotationTok}[1]{\textcolor[rgb]{0.56,0.35,0.01}{\textbf{\textit{#1}}}}
\newcommand{\AttributeTok}[1]{\textcolor[rgb]{0.77,0.63,0.00}{#1}}
\newcommand{\BaseNTok}[1]{\textcolor[rgb]{0.00,0.00,0.81}{#1}}
\newcommand{\BuiltInTok}[1]{#1}
\newcommand{\CharTok}[1]{\textcolor[rgb]{0.31,0.60,0.02}{#1}}
\newcommand{\CommentTok}[1]{\textcolor[rgb]{0.56,0.35,0.01}{\textit{#1}}}
\newcommand{\CommentVarTok}[1]{\textcolor[rgb]{0.56,0.35,0.01}{\textbf{\textit{#1}}}}
\newcommand{\ConstantTok}[1]{\textcolor[rgb]{0.00,0.00,0.00}{#1}}
\newcommand{\ControlFlowTok}[1]{\textcolor[rgb]{0.13,0.29,0.53}{\textbf{#1}}}
\newcommand{\DataTypeTok}[1]{\textcolor[rgb]{0.13,0.29,0.53}{#1}}
\newcommand{\DecValTok}[1]{\textcolor[rgb]{0.00,0.00,0.81}{#1}}
\newcommand{\DocumentationTok}[1]{\textcolor[rgb]{0.56,0.35,0.01}{\textbf{\textit{#1}}}}
\newcommand{\ErrorTok}[1]{\textcolor[rgb]{0.64,0.00,0.00}{\textbf{#1}}}
\newcommand{\ExtensionTok}[1]{#1}
\newcommand{\FloatTok}[1]{\textcolor[rgb]{0.00,0.00,0.81}{#1}}
\newcommand{\FunctionTok}[1]{\textcolor[rgb]{0.00,0.00,0.00}{#1}}
\newcommand{\ImportTok}[1]{#1}
\newcommand{\InformationTok}[1]{\textcolor[rgb]{0.56,0.35,0.01}{\textbf{\textit{#1}}}}
\newcommand{\KeywordTok}[1]{\textcolor[rgb]{0.13,0.29,0.53}{\textbf{#1}}}
\newcommand{\NormalTok}[1]{#1}
\newcommand{\OperatorTok}[1]{\textcolor[rgb]{0.81,0.36,0.00}{\textbf{#1}}}
\newcommand{\OtherTok}[1]{\textcolor[rgb]{0.56,0.35,0.01}{#1}}
\newcommand{\PreprocessorTok}[1]{\textcolor[rgb]{0.56,0.35,0.01}{\textit{#1}}}
\newcommand{\RegionMarkerTok}[1]{#1}
\newcommand{\SpecialCharTok}[1]{\textcolor[rgb]{0.00,0.00,0.00}{#1}}
\newcommand{\SpecialStringTok}[1]{\textcolor[rgb]{0.31,0.60,0.02}{#1}}
\newcommand{\StringTok}[1]{\textcolor[rgb]{0.31,0.60,0.02}{#1}}
\newcommand{\VariableTok}[1]{\textcolor[rgb]{0.00,0.00,0.00}{#1}}
\newcommand{\VerbatimStringTok}[1]{\textcolor[rgb]{0.31,0.60,0.02}{#1}}
\newcommand{\WarningTok}[1]{\textcolor[rgb]{0.56,0.35,0.01}{\textbf{\textit{#1}}}}
\usepackage{graphicx,grffile}
\makeatletter
\def\maxwidth{\ifdim\Gin@nat@width>\linewidth\linewidth\else\Gin@nat@width\fi}
\def\maxheight{\ifdim\Gin@nat@height>\textheight\textheight\else\Gin@nat@height\fi}
\makeatother
% Scale images if necessary, so that they will not overflow the page
% margins by default, and it is still possible to overwrite the defaults
% using explicit options in \includegraphics[width, height, ...]{}
\setkeys{Gin}{width=\maxwidth,height=\maxheight,keepaspectratio}
% Set default figure placement to htbp
\makeatletter
\def\fps@figure{htbp}
\makeatother
\setlength{\emergencystretch}{3em} % prevent overfull lines
\providecommand{\tightlist}{%
  \setlength{\itemsep}{0pt}\setlength{\parskip}{0pt}}
\setcounter{secnumdepth}{-\maxdimen} % remove section numbering

\title{Markovian Emotions}
\author{Jack Lovell}
\date{10/19/2020}

\begin{document}
\maketitle

\hypertarget{what-is-a-markov-chain}{%
\section{What is a Markov-Chain?}\label{what-is-a-markov-chain}}

Markovian processes can be summed up by a simple sentence: ``\emph{(a
markovian process) is a stochastic model describing a sequence of
possible events in which the probability of each event depends only on
the state attained in the previous event}'' By stochastic, we mean
random, and this idea can be visaulized thorugh the graphic displayed
below:

\includegraphics{https://upload.wikimedia.org/wikipedia/commons/thumb/2/2b/Markovkate_01.svg/220px-Markovkate_01.svg.png}

In this case we have states: \[A\] and \[E\]. In any case where we have
two states, we have 4 types of transitions: From \[E\] back to \[E\],
from \[E\] to \[A\], from \[A\] to \[E\], and from \[A\] to \[A\]. Each
transition can be modeled by how likely an agent is to transition from
one state to another, such as \[A\] → \[B\]. This probability is only
dependent on the agent's current state and can be summarized below:

\includegraphics{https://miro.medium.com/max/300/1*wAtMKljO3SMCqxRFf5mCTQ.png}
Where \[P~ij\] signifies the probability of the agent tranisiton from
state \[i\] to state \[j\], if our agent's state is the ranomd variable
\[X\]. Well, that probability is simply the conditional probability that
\[X\] will transition to state \[j\] given it is in state \[i\]. Thus,
the future state of an agent is only dependent on it's current state!

\hypertarget{using-a-markovian-process-to-model-emotions}{%
\section{Using a markovian process to model
emotions}\label{using-a-markovian-process-to-model-emotions}}

Markovian processes have been used to model a large number of natural
phenomena, ranging from physics and chemistry, to biology and social
sciences! Often the probabilities discussed above can be sumarized in a
transtion matrix as seen below:

\includegraphics{http://www.sosmath.com/matrix/markov/img2.gif}

This is also know as a \emph{stochastic matrix} and gives us loads of
insight into what our agent is doing! Trnasition matrices get's me
thinking, what about the transitions between emotions? Well luckily a
paper already published a ton of work on this!

\hypertarget{thorton-et-al.-2016-pnas}{%
\subsection{Thorton et al.~2016, PNAS}\label{thorton-et-al.-2016-pnas}}

In 2016, Mike Thorton and co. published a paper that aimed to quantify
how accurate people's mental models of emotion transitions are of actual
transitions between emotions. Even though the main objective of the
paper was not necessarily what we are interested in, they still have
valuable data for us to use (i.e.~the transition matricies from ground
truth studies)! So let's go ahead and take a look at them.

\begin{Shaded}
\begin{Highlighting}[]
\CommentTok{#data}
\KeywordTok{setwd}\NormalTok{(}\StringTok{"~/Desktop/report4/"}\NormalTok{)}
\NormalTok{gtrans <-}\StringTok{ }\KeywordTok{read.csv}\NormalTok{(}\StringTok{"study3/ground_transition_odds.csv"}\NormalTok{)}
\NormalTok{gtrans}
\end{Highlighting}
\end{Shaded}

\begin{verbatim}
##                X     Pride      Love      Hope Gratitude       Joy Satisfaction
## 1          Pride 2.6195146 1.1826766 1.5044767 2.0458326 1.2748912    1.3958793
## 2           Love 1.1902216 1.8893343 1.0880763 1.1187153 0.9480126    0.9166331
## 3           Hope 1.5099080 1.0851592 1.7807578 1.7694866 1.1015237    1.1356906
## 4      Gratitude 2.0374556 1.1289276 1.7622894 3.6523381 1.3409076    1.4485945
## 5            Joy 1.2454226 0.9492002 1.0862339 1.3317509 1.2347061    1.1684947
## 6   Satisfaction 1.3876857 0.9266632 1.1295195 1.4618843 1.1901851    1.3716074
## 7            Awe 2.2074205 1.2772894 1.6099203 2.8746463 1.3904013    1.4949611
## 8      Amusement 1.3378508 0.8998291 1.0531636 1.3837709 1.1052148    1.1291999
## 9      Alertness 1.1639355 0.7771538 0.9929578 1.2531730 0.9544156    1.0347552
## 10       Anxiety 0.6873633 0.8040981 0.7691114 0.6574367 0.5906420    0.5725594
## 11      Contempt 0.8030961 0.7599670 0.7108483 0.7348098 0.5550731    0.6083228
## 12       Offense 0.7939545 0.7613854 0.8287187 0.8100831 0.5732181    0.6127470
## 13         Guilt 0.7028675 0.9032297 0.7698041 0.8599494 0.5767446    0.5672657
## 14       Disgust 0.6112092 0.7448117 0.6887793 0.5853397 0.5153413    0.4982891
## 15          Fear 0.6886887 0.9219336 0.8564097 0.7532593 0.5150708    0.4957318
## 16 Embarrassment 0.7740014 0.8138827 0.8485610 0.8204342 0.6522738    0.6887075
## 17       Sadness 0.5534584 0.7448767 0.6545998 0.6468094 0.4819785    0.4664169
## 18         Anger 0.6849618 0.7805066 0.6973796 0.6641294 0.5441977    0.5378854
##          Awe Amusement Alertness   Anxiety   Contempt   Offense     Guilt
## 1  2.2908563 1.3492421 1.1584746 0.6884515  0.7722077 0.7261512 0.7288256
## 2  1.2781940 0.9171881 0.8045828 0.8156756  0.8164510 0.7755659 0.8853045
## 3  1.6870603 1.0860203 1.0054533 0.7458718  0.7790897 0.7381153 0.7466438
## 4  2.9036606 1.4172555 1.2310949 0.6386910  0.7474789 0.7750961 0.8550354
## 5  1.3761261 1.0837979 0.9538016 0.5797941  0.5672280 0.5955960 0.5720301
## 6  1.5101429 1.1148668 1.0399405 0.5695234  0.6438083 0.5869018 0.5770970
## 7  4.8581826 1.8498332 1.5067867 0.6384095  0.6912849 0.7290041 0.7708730
## 8  1.7952031 1.5788639 1.1314055 0.6509367  0.9044750 0.7429325 0.6815933
## 9  1.4880805 1.0948559 1.3423879 0.6916809  0.7804624 0.6958825 0.6795735
## 10 0.6053583 0.6609511 0.6950476 1.5330593  1.4416656 1.2785022 1.3405024
## 11 0.9381724 0.8700188 0.7804624 1.3281097 14.1546397 3.4832080 2.7450907
## 12 0.8181628 0.7228038 0.7297179 1.2585748  3.4236660 4.1297411 2.1073870
## 13 0.7517208 0.7233576 0.7094336 1.3443324  2.9625237 2.0785187 4.3749320
## 14 0.5953298 0.6302773 0.6230202 1.1364561  2.2572142 1.9782812 1.7685283
## 15 0.7291216 0.7109575 0.7019485 1.6029269  3.1795040 2.3814463 2.3973741
## 16 1.0789147 0.9498761 0.8174747 1.2196555  3.4271408 2.1234148 2.4308781
## 17 0.5826995 0.5530646 0.6236218 1.1413246  1.4856514 1.4916430 1.5071987
## 18 0.6600991 0.6628283 0.6510839 1.1714182  2.1538066 1.8896756 1.5477791
##      Disgust      Fear Embarrassment   Sadness     Anger
## 1  0.5969950 0.6685516     0.8369472 0.5534584 0.6416098
## 2  0.7429213 0.9420193     0.7952796 0.7394846 0.7340222
## 3  0.6531888 0.7863398     0.8691634 0.6116612 0.6171660
## 4  0.5911697 0.7606927     0.9724727 0.6100661 0.6081185
## 5  0.5090021 0.5301267     0.6697045 0.4874486 0.4985635
## 6  0.4852253 0.4977147     0.6703420 0.4552634 0.4861246
## 7  0.5430680 0.7339502     0.9112077 0.5888722 0.5535477
## 8  0.6366197 0.7075880     0.8835603 0.5483265 0.5658069
## 9  0.6093381 0.6697586     0.7597706 0.5936221 0.5979912
## 10 1.1341840 1.5816833     1.2744338 1.1158960 1.1547876
## 11 2.1411289 3.2343231     3.0146146 1.4716359 2.1243024
## 12 2.0714412 2.3930915     2.1840838 1.5229049 1.9648865
## 13 1.7385109 2.4345839     2.5539605 1.5126349 1.5163086
## 14 2.4074648 1.7658618     1.5741794 1.5692078 1.7609665
## 15 1.7507258 4.9747735     2.4204535 1.5556105 1.6445710
## 16 1.5449739 2.3459780     4.1530463 1.2345009 1.3093949
## 17 1.5421191 1.6063221     1.2297406 1.8722760 1.4767995
## 18 1.7854055 1.7340126     1.3528187 1.4959787 2.3479131
\end{verbatim}

Ok great! Now we have a set of emotions and their transitions! They are
normed to their frequency, for various reasons but this may give us
trouble, so let's come back to that next time. For now, we can estimate
and plot a markov chain!

\begin{Shaded}
\begin{Highlighting}[]
\ControlFlowTok{if}\NormalTok{(}\OperatorTok{!}\KeywordTok{require}\NormalTok{(markovchain)) }\KeywordTok{install.packages}\NormalTok{(}\StringTok{"markovchain"}\NormalTok{); }\KeywordTok{require}\NormalTok{(markovchain)}
\end{Highlighting}
\end{Shaded}

\begin{verbatim}
## Loading required package: markovchain
\end{verbatim}

\begin{verbatim}
## Package:  markovchain
## Version:  0.8.5-2
## Date:     2020-09-07
## BugReport: https://github.com/spedygiorgio/markovchain/issues
\end{verbatim}

\begin{Shaded}
\begin{Highlighting}[]
\CommentTok{#now lets call a new markovchain instance}
\NormalTok{mcEmotion <-}\StringTok{  }\KeywordTok{new}\NormalTok{(}\StringTok{"markovchain"}\NormalTok{, }\DataTypeTok{states =}\NormalTok{ gtrans[,}\DecValTok{1}\NormalTok{], }\DataTypeTok{byrow=}\NormalTok{T, }\DataTypeTok{transitionMatrix=}\KeywordTok{as.matrix}\NormalTok{(gtrans[,}\DecValTok{2}\OperatorTok{:}\DecValTok{19}\NormalTok{]), }\DataTypeTok{name =} \StringTok{"EmoMarkov"}\NormalTok{)}
\end{Highlighting}
\end{Shaded}

\begin{verbatim}
## Error in validObject(.Object): invalid class "markovchain" object: Error! Rows of transition matrix do not some one
\end{verbatim}

It doesn't work! God dangit!!! Why not? Well it is because the sums of
the rows do not add to one, this is because they are normalized to the
frequency! This will take some effort to fix, and I do not have the time
to do so. So let's save it for next week, right now, we can use the
example in the \texttt{markovchain} package to learn some more about how
to code a Markov Chain in R.

\begin{Shaded}
\begin{Highlighting}[]
\ControlFlowTok{if}\NormalTok{(}\OperatorTok{!}\KeywordTok{require}\NormalTok{(markovchain)) }\KeywordTok{install.packages}\NormalTok{(}\StringTok{"markovchain"}\NormalTok{); }\KeywordTok{require}\NormalTok{(markovchain)}
\NormalTok{weatherStates <-}\StringTok{ }\KeywordTok{c}\NormalTok{(}\StringTok{"sunny"}\NormalTok{, }\StringTok{"cloudy"}\NormalTok{, }\StringTok{"rain"}\NormalTok{)}
\NormalTok{byRow <-}\StringTok{ }\OtherTok{TRUE}
\NormalTok{weatherMatrix <-}\StringTok{ }\KeywordTok{matrix}\NormalTok{(}\DataTypeTok{data =} \KeywordTok{c}\NormalTok{(}\FloatTok{0.70}\NormalTok{, }\FloatTok{0.2}\NormalTok{, }\FloatTok{0.1}\NormalTok{,}
                                 \FloatTok{0.3}\NormalTok{, }\FloatTok{0.4}\NormalTok{, }\FloatTok{0.3}\NormalTok{,}
                                 \FloatTok{0.2}\NormalTok{, }\FloatTok{0.45}\NormalTok{, }\FloatTok{0.35}\NormalTok{), }\DataTypeTok{byrow =}\NormalTok{ byRow, }\DataTypeTok{nrow =} \DecValTok{3}\NormalTok{, }
                                \DataTypeTok{dimnames =} \KeywordTok{list}\NormalTok{(weatherStates, weatherStates))}
\NormalTok{mcWeather <-}\StringTok{ }\KeywordTok{new}\NormalTok{(}\StringTok{"markovchain"}\NormalTok{, }\DataTypeTok{states =}\NormalTok{ weatherStates, }\DataTypeTok{byrow =}\NormalTok{ byRow,}
                  \DataTypeTok{transitionMatrix =}\NormalTok{ weatherMatrix, }\DataTypeTok{name =} \StringTok{"Weather"}\NormalTok{)}
\NormalTok{mcWeather}
\end{Highlighting}
\end{Shaded}

\begin{verbatim}
## Weather 
##  A  3 - dimensional discrete Markov Chain defined by the following states: 
##  sunny, cloudy, rain 
##  The transition matrix  (by rows)  is defined as follows: 
##        sunny cloudy rain
## sunny    0.7   0.20 0.10
## cloudy   0.3   0.40 0.30
## rain     0.2   0.45 0.35
\end{verbatim}

Great! By doing so we get a sweet transition matrix from a discrete
Markov Chain with 3 dimensions, now let's visualize it!

\begin{Shaded}
\begin{Highlighting}[]
\KeywordTok{plot}\NormalTok{(mcWeather)}
\end{Highlighting}
\end{Shaded}

\includegraphics{report4_files/figure-latex/unnamed-chunk-3-1.pdf}
Aweosme! This looks so great! Not ideal that we were unable to do this
for emotions, but I will be stopping by office hours to discuss an
approach to compute the transition probabilities for each state. That's
all for now bye-bye!!

\end{document}
